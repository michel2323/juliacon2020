\documentclass[10pt]{beamer}
\usepackage{etex}
% \usepackage{subfig}
\usepackage[english]{babel}
\usepackage[utf8]{inputenc}
\usepackage[T1]{fontenc}
\usepackage{lmodern}
\usepackage{caption, subcaption}
\usepackage{bm}

% AMSLaTeX packages
\usepackage{amsthm,amsmath,amsfonts}
\usepackage[algoruled]{algorithm2e}

\usetheme{default}
\useoutertheme{default}
% we want to use images
\usepackage{graphicx,hyperref,grffile}

% table relates packages
\usepackage{booktabs,multirow}
\usepackage{realboxes}

\usepackage{xcolor}
\usepackage{adjustbox}
\usepackage{ulem}
%\usepackage{subfigure}
%\usepackage[numbers]{natbib}
%\usepackage{natbib}
\usepackage[style=authortitle]{biblatex}
\addbibresource{presentation.bib}
% \setbeamercolor{bibliography entry note}{fg=red}
%\usepackage{todonotes}
%\presetkeys{todonotes}{inline}{}
%\usepackage{multimedia}
%\usepackage{bibentry}
%\nobibliography*
%\let\labelindent\relax
%\usepackage{enumitem}
%\usepackage{pst-grad} % For gradients
%\usepackage{pst-plot}
%\usepackage{pstricks-add}
\usepackage{tikz}
\usepackage{tikzscale}
\usepackage{pgfplots}
\usepackage{xspace}
% color cycle from https://colorbrewer1.org/?type=qualitative&scheme=Dark2&n=6
\usepgfplotslibrary{patchplots}
 \definecolor{cb1}{RGB}{27,158,119}
 \definecolor{cb2}{RGB}{217,95,2}
 \definecolor{cb3}{RGB}{117,112,179}
 \definecolor{cb4}{RGB}{231,41,138}
 \definecolor{cb5}{RGB}{102,166,30}
 \definecolor{cb6}{RGB}{230,171,2}
\pgfplotscreateplotcyclelist{cbw}{%
    ultra thick, cb1, densely dashed, mark=*\\%
    ultra thick, cb1, mark=*\\%
    ultra thick, cb2, densely dashed, mark=square*\\%
    ultra thick, cb2, mark=square*\\%
    ultra thick, cb3, densely dashed, mark=triangle*\\%
    ultra thick, cb3, mark=triangle*\\%
    ultra thick, cb4, densely dashed, mark=star\\%
    ultra thick, cb4, mark=star\\%
    ultra thick, cb5, densely dashed, mark=diamond*\\%
    ultra thick, cb5, mark=diamond*\\%
}
\pgfplotscreateplotcyclelist{cbww}{%
    ultra thick, cb1, fill, mark=none\\%
    ultra thick, cb2, fill, mark=none\\%
    ultra thick, cb3, fill, mark=none\\%
    ultra thick, cb6, fill, mark=none\\%
    ultra thick, cb6\\%
    ultra thick, cb6, densely dashed\\%
    ultra thick, cb5, densely dashed, mark=diamond*\\%
    ultra thick, cb5, mark=diamond*\\%
}
\usepackage[off]{pstricks,auto-pst-pdf} % when switched to on, don't forget to include
                               % the aforementioned pstricks packages.
\pgfplotsset{compat=1.15}
\pgfplotstableread{P.dat}\matrixP
\pgfplotstableread{M.dat}\matrixM
\pgfplotstableread{J.dat}\matrixJ
\pgfplotstableread{N.dat}\matrixN
\def\nrows{52}
\def\ncols{52}
\newcommand{\BiCGSTAB}{\textsc{BiCGSTAB}\xspace}
\newcommand{\GMRES}{\textsc{GMRES}\xspace}
\newcommand{\RGMRES}[1]{\textsc{GMRES}(#1)\xspace}
\newcommand{\DQGMRES}{\textsc{DQGMRES}\xspace}
% Changes the color of bullets and other fonts to match the NormalANLBlue template
\definecolor{anlpresentationblue}{RGB}{127 160 195}
\setbeamercolor{structure}{fg=anlpresentationblue!150} % for slide titles, etc
\setbeamercolor{block title}{fg=white,bg=anlpresentationblue} % For blocks
\setbeamercolor{block body}{fg=black,bg=anlpresentationblue!20} % For blocks

% Force revisiting of table of contents every time a new section is declared
\AtBeginSection[]
{
   \begin{frame}
       \frametitle{Outline}
       \tableofcontents[currentsection]
   \end{frame}
}

\setlength\arraycolsep{1.4pt}% some length
\definecolor{darkgreen}{HTML}{3C8031}
%\definecolor{darkgreen}{HTML}{2D2F92}
%\definecolor{darkgreen}{HTML}{F58137}
\newcommand{\vect}[1]{\boldsymbol{#1}}

%gets rid of navigation symbols
\setbeamertemplate{navigation symbols}{}
\setbeamertemplate{itemize items}[circle]

%gets rid of bottom navigation bars
\setbeamertemplate{footline}[page number]{}
\setbeamertemplate{headline}{}
\hypersetup{
    colorlinks=true,
    citecolor = blue
}
%<your other options\dots>,

%\usepackage{cmbright} % A font that I like
% \usepackage[T1]{fontenc}

% \renewcommand{\seriesdefault}{sb} % Changes the font to semibold

\usebackgroundtemplate{\includegraphics[width=\paperwidth]{NormalANLBlue}}

\title[]{
Domain Decomposition for Unstructured Nonlinear Programming on Parallel Vector Architectures \\
}
\author[]{Michel Schanen*, Daniel Adrian Maldonado, François Pacaud, Alexis Montoison, Mihai Anitescu}
\subtitle{}
\institute[ANL/MCS]{Argonne National Laboratory\\ Mathematics and Computer Science
Division}
\date{Dec 12, 2020}

\usepackage{tikz,tikz-cd}
\usepackage{pgfplots,pgfplotstable}
\usetikzlibrary{pgfplots.groupplots}
\usetikzlibrary{positioning}
\usetikzlibrary[shapes,arrows,trees]
\usetikzlibrary{matrix,decorations.pathmorphing}
% \usepackage{epstopdf}
\usepackage{tensor}
\usepackage[usecolors=true, usebox=true, charsperline=65]{jlcode}
\addtobeamertemplate{footnote}{}{ \vspace{2ex}}

\newcommand{\Var}[1]{\mathrm{Var}\left \{ #1\right\}} % Variance
\newcommand{\vol}[1] {\operatorname{vol}\left( #1 \right)}
\newcommand{\E}[1]{\operatorname{\mathbb{E}}\left[ #1\right]} % Expected value
\renewcommand{\P}[1]{\operatorname{\mathbb{P}}\left[ #1\right]} % Expected value
\newcommand{\norm}[1]{\left\| {#1} \right\|}
\newcommand{\argmax}{\operatornamewithlimits{arg\,max}}
\newcommand{\argmin}{\operatornamewithlimits{arg\,min}}
\newcommand{\jeffa}{\phantom{\sum_{i = 1}^{m}\tilde{f}\mathcal{N}(0,\sigma^2)}}
\newtheorem{proposition}[theorem]{Proposition}
\newcommand\Perms[2]{\tensor[_{#1}]P{_{#2}}}
\newcommand{\minimize}{\operatornamewithlimits{minimize}}
\newcommand{\tu}[1]{\textup{#1}}
\newtheorem{assumption}[theorem]{Assumption}
%\newcommand{\adj}[1]{{#1_{(1)}}}
%\newcommand{\tlm}[1]{{#1^{(1)}}}
\newcommand{\adj}[1]{\bar{#1}}
\newcommand{\tlm}[1]{\dot{#1}}
%\newcommand{\tlm}[1]{{#1^{(1)}}}
% \setbeameroption{show notes}
% \setbeameroption{show only notes}
\newcommand{\mynode}[2]{%
%\pspolygon(#1,#2)(#3,#4)(! 0.5 dup add #1 sub #2)(! #3 #2 dup add #4 sub)
%\psframe(! 0.5 0.5 add #1 sub 1 1 add)(1 ,6)
\psframe(! #1 0.5  sub #2 0.5 sub)(! #1 0.5  add #2 0.5 add)
%\psline(! #1 0.5 sub #2)(! #1 0.5 add #2)
%\rput(! #1 -0.25 #2 add){CPU}
%\rput(! #1 0.25 #2 add){Mem.}
}

% Some macros for Taylor series
\newcommand{\sumone}[1]{\ensuremath{\sum_{#1 = 1}^n}}
\newcommand{\sumtwo}[2]{\ensuremath{\sum_{#1,#2 = 1}^n}}
\newcommand{\sumthree}[3]{\ensuremath{\sum_{#1,#2,#3 = 1}^n}}
\newcommand{\sumfour}[4]{\ensuremath{\sum_{#1,#2,#3,#4 = 1}^n}}
\newcommand{\D}[0]{\mathrm{D}}

\newcommand{\dpart}[2]{\dfrac{\partial #1}{\partial #2}}
\newcommand{\deriv}[2]{\dfrac{d #1}{d #2}}
\newcommand{\grad}[2]{\nabla_{#2}{#1}}
\newcommand{\hess}[2]{\dfrac{d^2 #1}{d #2}}
\newcommand{\dhess}[2]{\dfrac{\partial^2 #1}{\partial #2}}

\newcommand{\partone}[2]{\D_{#2} g_{#1}}
%\newcommand{\parttwo}[3]{H^{#1}_{#2#3}}
\newcommand{\parttwo}[3]{\D^2_{#2 #3} g_{#1}}
%\newcommand{\partthree}[4]{T^{#1}_{#2#3#4}}
\newcommand{\partthree}[4]{\D^3_{#2 #3 #4} g_{#1}}
\newcommand{\bigo}[1]{\mathcal{O}\left( #1 \right)}
%\newcommand{\vect}[1]{\boldsymbol{#1}}
%\newcommand{\E}[1]{\mathbb{E}\left[ #1 \right]}
\newcommand{\REAL}{\mathbb{R}}

\usepackage{array}
\newcolumntype{x}[1]{>{\centering\arraybackslash\hspace{0pt}}p{#1}}
\setbeameroption{hide notes}

\begin{document}
\setbeamertemplate{footline}{}
{
\usebackgroundtemplate{\includegraphics[width=\paperwidth]{TitleANLBlue}}
\frame{\titlepage}
}

% \setbeamertemplate{footline}[]{\insertframenumber of \inserttotalframenumber}
\setbeamertemplate{footline}{
  \begin{beamercolorbox}{footlinecolor}
    \hfill
    \begin{minipage}[c]{3cm}
        \tiny{{\color{anlpresentationblue} \hfill \insertframenumber{} of \inserttotalframenumber} }
    \end{minipage}      
    \begin{minipage}[c]{10cm}
       {\color{white} .}
    \end{minipage}      
   \end{beamercolorbox}
}

% DO NOT COMPILE THIS FILE DIRECTLY!
% This is included by the other .tex files.
\begin{frame}
  \frametitle{Upcoming Supercomputers}
    \begin{center}
      \includegraphics[width=.25\textwidth]{./figures/ecp} \\
      % \includegraphics[width=.75\textwidth]{./figures/go} 
    \end{center}
  \begin{columns}[T]
    \begin{column}{0.49\textwidth}
      \begin{center}
        {\bf Aurora}\\
        \includegraphics[width=0.75\textwidth]{./figures/aurora}
      \end{center}
    \end{column}
    \begin{column}{0.49\textwidth}
      \begin{center}
        {\bf Frontier}\\
        \includegraphics[width=0.75\textwidth]{./figures/frontier}
      \end{center}
    \end{column}
  \end{columns}
  \begin{columns}[T]
    \begin{column}{0.49\textwidth}
    \begin{itemize}
      \item Intel’s Xe compute architecture.
      \item $>$ 1 exaflops
    \end{itemize}
    \end{column}
    \begin{column}{0.49\textwidth}
      \begin{itemize}
        \item AMD EPYC processors and Radeon Instinct GPU
        \item 1.5 exaflops
      \end{itemize}
    \end{column}
  \end{columns}
\end{frame}

\begin{frame}
  \frametitle{Distributed Parallel Security Constrainted Optimal Power Flow}
  \begin{center}
    \includegraphics[width=0.65\textwidth]{figures/twostageopt}
  \end{center}
  \begin{itemize}
    \item Two-stage optimization using distributed parallelism with Schur complement decomposition
    \item PIPS-NLP \footnote{https://github.com/Argonne-National-Laboratory/PIPS} implements Schur complement for two-stage optimization
    \item HiOp\footnote{https://github.com/LLNL/hiop} mixed dense-sparse solver 
    \item Problem: On-node performance with GPUs
  \end{itemize}
\end{frame}

\begin{frame}
  \frametitle{Overview}
  \begin{itemize}
    \item Motivation: Interior-point for GPUs is hard
    \item Solution: Move toward feasible direction method using iterative linear solvers
    \item Requirement: Fast implementation of reduced gradient/Hessian computation
    \item In this talk: Fast reduced gradient computation on GPUs
    \item Future goal: Reduced Hessian optimization algorithm for alternate current optimal power flow (ACOPF)
  \end{itemize}
\end{frame}

\begin{frame}
  \frametitle{First-Order Reduced Method}
  State $\bm{x}$, control $\bm{u}$, and function $h: \REAL^{n} \times \REAL^{m} \to \REAL^{p}$
  \begin{equation}
    \underline{x} \leq \bm{x} \leq \overline{x} \;, \quad
    \underline{u} \leq \bm{u} \leq \overline{u} \;, \quad
    h(\bm{x}, \bm{u}) \leq 0 \;.
  \end{equation}
  Optimization Problem:
  \begin{equation}\label{eq:globalproblem}
    \begin{array}{r@{\;\;}l}
      \min_{\bm{x}, \bm{u}} & f(\bm{x}, \bm{u}) \\
      \text{s.t.} & g(\bm{x}, \bm{u}) = 0 \\
                  & h(\bm{x}, \bm{u}) \leq 0 \;,
                  \quad \underline{u} \leq \bm{u} \leq \overline{u}  \;.
    \end{array}
  \end{equation}
  Augmented Lagrangian $\mathcal{L}$:
  \begin{equation}
    \mathcal{L}(\bm{x}, \bm{u}, \lambda) = f(\bm{x}, \bm{u}) + \lambda^\top g(\bm{x}, \bm{u}) \; ,
  \end{equation}
  Solver for multipliers $\lambda$:
  \begin{equation}
    \dpart{g}{x}^\top \lambda = -\dpart{f}{x}  \; .
  \end{equation}
  Compute reduced gradient:
  \begin{equation}
    \grad{f}{u} = \dpart{f}{u} + \dpart{g}{u}^\top\lambda \; .
  \end{equation}

\end{frame}

% \begin{frame}
%   \frametitle{Complex Networks}
%   \includegraphics[width=\textwidth]{figures/complexn}
%   \begin{itemize}
%     \item Examples: traffic, Internet, pipelines, power grid
%     \item Connection pattern is irregular but not random
%   \end{itemize}
% \end{frame}

\begin{frame}[fragile]
  \frametitle{NLP: Alternate Current Optimal Power Flow}
  \begin{itemize}
    \item {\bf Objective}
    \begin{itemize}
      \item Generation $P_g$ and its cost $c$ at generators $g_i$:
      $ \minimize \sum^G_{i=1} c_i(P_{g_i})$, where $c_i$ is the cost of generator $g_i$
    \end{itemize}
    \item {\bf Constraints}
    \begin{itemize}
      \item Kirchhoff's law: What flows in must flow out (nonlinear, non-convex in ACOPF)
      \begin{align*}
        V_k e^{-j\theta_k} & \sum^{N}_{m=0} (G_{km} + jB_{km})V_m e^{j\theta_m} = P_k - j Q_k,\ k = 1, \dots, N \\
        \text{where}\\
        V_k &\text{ voltage magnitude at node } k\\
        \theta &\text{ voltage angle at node } k\\
        G_{km} + jB_{km}& \text{ element of nodal admittance matrix}\\
        P_k , Q_k &\text{ net real and reactive power entering and leaving node } k
      \end{align*}
      \item Line limits defined through voltage angles $\theta$ at the buses $m$ and $n$:
      $$ \theta^{min}_{nm} \leq \theta_n - \theta_m \leq \theta^{max}_{nm}$$
    \end{itemize}
  \end{itemize}
\end{frame}

\begin{frame}[fragile]
  \frametitle{Use Interior-Point for NLP in Full Space}
  {\bf Solve}
  \begin{align*}
  &\minimize f(x)\\ 
  \text{with}&\\
  &g(x) \geq 0, \ i=1,\dots, m \\
  &c(x) = 0, \ j=1,\dots, l \\
  \end{align*}
  {\bf using Newton and barrier functions}
  \begin{align*}
  &\minimize \phi_\mu := f(x) - \mu \sum^m_{i=1} \ln g(x)\\ 
  \text{with}&\\
  &c(x) = 0, \ j=1,\dots, l 
  \end{align*}
  \begin{itemize}
    \item Barrier functions exacerbate ill-conditioning
  \end{itemize}
\end{frame}

\begin{frame}[fragile]
  \frametitle{Interior-Point and GPUs}
  {\bf Current State}
  \begin{itemize}
    \item De facto standard solver: Ipopt
    \item Very ill-conditioned system (up $1e^{16}$)
    \item Requires indefinite sparse direct inertia revealing solver
    \item Ipopt only supports direct solver (exception: Pardiso)
    \item default MUMPS, preferred are HSL libary MA27, MA57 
    \item \alert{GPUs want iterative linear solvers}
  \end{itemize}
\end{frame}

\begin{frame}
\frametitle{First-Order Reduced Method}
  State $\bm{x}$, control $\bm{u}$, and function $h: \REAL^{n} \times \REAL^{m} \to \REAL^{p}$
  \begin{equation}
    \underline{x} \leq \bm{x} \leq \overline{x} \;, \quad
    \underline{u} \leq \bm{u} \leq \overline{u} \;, \quad
    h(\bm{x}, \bm{u}) \leq 0 \;.
  \end{equation}
  Optimization Problem:
  \begin{equation}\label{eq:globalproblem}
    \begin{array}{r@{\;\;}l}
      \min_{\bm{x}, \bm{u}} & f(\bm{x}, \bm{u}) \\
      \text{s.t.} & \alert{g(\bm{x}, \bm{u}) = 0 }\\
                  & h(\bm{x}, \bm{u}) \leq 0 \;,
                  \quad \underline{u} \leq \bm{u} \leq \overline{u}  \;.
    \end{array}
  \end{equation}
  Augmented Lagrangian $\mathcal{L}$:
  \begin{equation}
    \mathcal{L}(\bm{x}, \bm{u}, \lambda) = f(\bm{x}, \bm{u}) + \lambda^\top g(\bm{x}, \bm{u}) \; ,
  \end{equation}
  Solver for multipliers $\lambda$:
  \begin{equation}
    \dpart{g}{x}^\top \lambda = -\dpart{f}{x}  \; .
  \end{equation}
  Compute reduced gradient:
  \begin{equation}
    \grad{f}{u} = \dpart{f}{u} + \dpart{g}{u}^\top\lambda \; .
  \end{equation}

\end{frame}



\begin{frame}[fragile]
  \frametitle{\sout{Optimal} Power Flow for Reduced Space}
  \begin{itemize}
    \item {\bf \sout{Objective}}
    \begin{itemize}
      \item \sout{Generation cost at generators:
      $ \minimize \sum^G_{i=1} c_i(P_{g_i})$, where $c_i$ is the cost of generator $g_i$}
    \end{itemize}
    \item {\bf Constraints}
    \begin{itemize}
      \item Kirchhoff's law: What flows in must flow out (nonlinear, non-convex in ACOPF)
      \begin{align*}
        V_k e^{-j\theta_k} & \sum^{N}_{m=0} (G_{km} + jB_{km})V_m e^{j\theta_m} = P_k - j Q_k,\ k = 1, \dots, N \\
        \text{where}\\
        V_k &\text{ voltage magnitude at node } k\\
        \theta &\text{ voltage angle at node } k\\
        G_{km} + jB_{km}& \text{ element of nodal admittance matrix}\\
        P_k , Q_k &\text{ net real and reactive power entering and leaving node } k
      \end{align*}
      \item \sout{Line limits: $ \theta^{min}_{nm} \leq \theta_n - \theta_m \leq \theta^{max}_{nm}$}
    \end{itemize}
  \end{itemize}
\end{frame}

\begin{frame}[fragile]
  \frametitle{Power Flow}
  {\bf Nonlinear equations}
  \begin{itemize}
      \item Kirchhoff's law: What flows in must flow out (nonlinear, non-convex in ACOPF)
      \begin{align*}
        V_k e^{-j\theta_k} & \sum^{N}_{m=0} (G_{km} + jB_{km})V_m e^{j\theta_m} = P - jQ,\ k = 1, \dots, N \\
        \text{where}\\
        V_k &\text{ voltage magnitude at node } k\\
        \theta &\text{ voltage angle at node } k\\
        G_{km} + jB_{km}& \text{ element of nodal admittance matrix}\\
        P, Q &\text{ net real and reactive power entering and leaving node } k
      \end{align*}
      \item Use Newton-Raphson to solve nonlinear equations
      \item Reference implementation: MATPOWER in Matlab
  \end{itemize}
\end{frame}

\begin{frame}
  \frametitle{Implementation}
  {\bf Goals}
  \begin{itemize}
    \item Compute Jacobian using automatic differentiation
    \item Implement a preconditioner
    \item Implement a Krylov method
    \item No computation on the host in main loop, no data transfer
    \item Entirely written in Julia
  \end{itemize}
\end{frame}

% \begin{frame}
%   \frametitle{}
%   \centering
%   {\Huge Automatic Differentiation}
% \end{frame}

% \begin{frame}[fragile]
%   \frametitle{Derivatives}
%   {\bf Newton-Raphson}
%   \begin{minipage}{0.8\textwidth}
%     \begin{lstlisting}
%     go = true
%     while(go)
%       dx .= jacobian(x)\f(x)
%       x  .= x .- dx
%       go = norm(f(x)) < tol ? true : false
%     end
%     \end{lstlisting}
%    \end{minipage}
%   {\bf Automatic Differentiation}
%   \begin{itemize}
%     \item \lstinline{F = f(x)} $\rightarrow$ \alert{\lstinline{J = jacobian(x)}}\footnote{\cite{RevelsLubinPapamarkou2016}}
%     \item Two modes: \Colorbox{green}{Adjoint} or \Colorbox{red}{tangent}
%     \item \Colorbox{green}{\lstinline{adj(x, y) = (J(x))'*y)}}, \Colorbox{red}{\lstinline{tgt(x,d) = J(x)*d}} 
%     \item \Colorbox{green}{\lstinline{size(x) >> size(F)}} or \Colorbox{red}{\lstinline{size(x) <= size(F)}}
%     \item number of buses $\propto$ \lstinline{size(x) = size(F)}
%   \end{itemize}
% \end{frame}

% \begin{frame}
%   \frametitle{Jacobian Coloring}
%   \begin{center}
%     \begin{figure}
%       \includegraphics[width=0.65\textwidth]{figures/compression}
%       \caption{Jacobian compression}
%     \end{figure}
%   \end{center}
%   \begin{itemize}
%     \item Using greedy algorithm in \lstinline{SparseDiffTools.jl}
%     \item To get full Jacobian, one has to do $c$ times \lstinline{tgt(x,d) = J(x)*d}
%     \item Complexity for Jacobian goes from $\bigo{n} \times cost(f)$ to $\bigo{c} \times cost(f)$
%   \end{itemize}
% \end{frame}

% \begin{frame}
%   \frametitle{AD on GPUs in Julia}
%   \begin{center}
%       \includegraphics[width=0.75\textwidth]{figures/F}
%   \end{center}
% \end{frame}
% \begin{frame}
%   \frametitle{AD on GPUs in Julia}
%   \begin{center}
%     \lstinline{F = T(undef, n)}
%   \end{center}
%   \begin{itemize}
%     \item Float vector: \lstinline{T = Vector\{Float64\}}
%     \item Arbitrary precision vector: \lstinline{T = Vector\{BigFloat\}}
%     \item First-order tangent: \lstinline{t1s\{N\} =  ForwardDiff.Dual\{Nothing,Float64, N\} where N}
%     \item Second-order tangent: \lstinline{t2s\{M,N\} =  ForwardDiff.Dual\{Nothing,t1s\{N\}, M\} where M, N}
%     \item First-order tangent vector: \lstinline{T = Vector\{t1s\{N\}\}}
%     \item First-order tangent GPU vector: \lstinline{T = CuVector\{t1s\{N\}\}}
%     \item Relying on \lstinline{CUDA.jl} \footnote{\cite{besard2018juliagpu}} \footnote{\cite{besard2019prototyping}}
%     \item \alert{broadcast operator \lstinline{x .= a .* b}}
%   \end{itemize}
% \end{frame}

% \begin{frame}
%   \frametitle{AD on GPUs in Julia}
%   \begin{columns}[T]
%     \begin{column}{0.35\textwidth}
%       \begin{center}
%         \vspace{0.2cm}
%         \includegraphics[width=\linewidth]{figures/compression}
%       \end{center}
%     \end{column}
%     \begin{column}{0.6\textwidth}
%       \begin{center}
%         \includegraphics[width=\linewidth]{figures/simd}
%       \end{center}
%     \end{column}
%   \end{columns}
%   \begin{center}
%   \end{center}
%   {\bf SIMD over sparsity colors}
%   \begin{itemize}
%     \item Change from \lstinline{F = f(x)} to \lstinline{J = f(x)} on GPUs is a type change from \lstinline{Vector\{Float64\}} to \lstinline{CuVector\{t1s\{c\}\}}
%   \end{itemize}
% \end{frame}

\begin{frame}
  \frametitle{}
  \centering
  {\Huge Preconditioned BiCGSTAB}
\end{frame}

\begin{frame}
  \frametitle{Linear solver}
    \begin{figure}
      \includegraphics[width=0.75\textwidth]{figures/gmresbicgstab}
    \end{figure}
  \begin{itemize}
    \item Best for GPUs: Iterative solver
    \item GMRES and BiCGSTAB \footnote{\cite{sleijpen1993bicgstab}} in \lstinline{IterativeSolvers.jl}, no GPU support 
    \item GMRES in \lstinline{Krylov.jl}, no BiCGSTAB 
    \item Wrote a naive implementation of BiCGSTAB \footnote{\cite{bicgstabVorst}}
  \end{itemize}
\end{frame}

\begin{frame}
  \frametitle{Preconditioner}
  {\bf Setup}
  \begin{itemize}
    \item Create a partitioning (METIS)
  \end{itemize}
  {\bf Update P}
  \begin{tabular}{m{0.15\textwidth}m{0.05\textwidth}m{0.3\textwidth}m{0.05\textwidth}m{0.3\textwidth}}
        \includegraphics[width=0.15\textwidth]{./figures/compressed}
    &
    {\Huge $\rightarrow$}
    &
        \includegraphics[width=0.3\textwidth]{figures/gpublocks}
    &
    {\Huge $\rightarrow$}
    &
        \includegraphics[width=0.3\textwidth]{./figures/csr}
    \\
  \end{tabular}
    \begin{enumerate}
      \item Read dense compressed Jacobian into dense Jacobi blocks
      \item Batch inversion of dense blocks using CUBLAS
      \item Update sparse matrix P from dense Jacobi blocks
    \end{enumerate}
  {\bf Code size}
  \begin{itemize}
    \item 200 lines of code for BOTH CPU and GPU implementation
  \end{itemize}
\end{frame}

% \begin{frame}
%   \frametitle{Preconditioner}
%   \includegraphics[width=0.95\textwidth]{figures/preconditioner}
% \end{frame}



% \begin{frame}[fragile]
%   \frametitle{Linear Solver}
%   \begin{center}
%    \includegraphics[width=0.7\textwidth]{figures/bicgstab.png}
%   \end{center}
% \end{frame}

% \begin{frame}
%   \frametitle{SIMD on GPUs in Julia}
%   \begin{center}
%       \includegraphics[width=0.75\textwidth]{figures/F}
%   \end{center}
% \end{frame}
% \begin{frame}
%   \frametitle{AD on GPUs in Julia}
%   \begin{center}
%     \lstinline{F = T(undef, n)}
%   \end{center}
%   \begin{itemize}
%     \item Float vector: \lstinline{T = Vector\{Float64\}}
%     \item Arbitrary precision vector: \lstinline{T = Vector\{BigFloat\}}
%     \item First-order tangent: \lstinline{t1s\{N\} =  ForwardDiff.Dual\{Nothing,Float64, N\} where N}
%     \item Second-order tangent: \lstinline{t2s\{M,N\} =  ForwardDiff.Dual\{Nothing,t1s\{N\}, M\} where M, N}
%     \item First-order tangent vector: \lstinline{T = Vector\{t1s\{N\}\}}
%     \item First-order tangent GPU vector: \lstinline{T = CuVector\{t1s\{N\}\}}
%     \item Relying on \lstinline{CUDA.jl} \footnote{\cite{besard2018juliagpu}} \footnote{\cite{besard2019prototyping}}
%     \item \alert{broadcast operator \lstinline{x .= a .* b}}
%   \end{itemize}
% \end{frame}

% \begin{frame}
%   \frametitle{}
%   \centering
%   {\Huge Results}
% \end{frame}

\begin{frame}[fragile]
  \frametitle{ExaPF.jl}
  % \begin{lstlisting}
  %   pkg> instantiate
  %   julia> target = "cuda" 
  %   julia> include("examples/pf.jl") 
  %   julia> datafile = "test/case14.raw"
  %   julia> sol, conv, res = pf(datafile, 8, "bicgstab") # slow
  %   julia> sol, conv, res = pf(datafile, 8, "bicgstab")
  % \end{lstlisting}
  \begin{itemize}
    \item \url{https://exanauts.github.io/}
    \item Julia for Summit builds
    \item Other research software
    \item \url{https://github.com/exanauts/ExaPF.jl}
  \end{itemize}
\end{frame}

\begin{frame}[fragile]
  \frametitle{ExaPF.jl}
  {\bf 30,000 bus system}
  % \begin{lstlisting}
  %   julia> datafile = "GO-Data/.../Network_30R-025/.../case.raw"
  %   julia> sol, conv, res = pf(datafile, 1000, "bicgstab")
  % \end{lstlisting}
  \begin{itemize}
    \item \lstinline{size(J)} = 57,721 $\times$ 57,721 
    \item Block Jacobi size: 59 $\times$ 59 $\approx$ 25 MB
    \item Total GPU memory usage: 5 GB
    \item Number of Jacobian colors: 29
    \item 4 Newton iterations with $tol = 1e^{-6}$, total BiCGSTAB iterations: 3,752 
    \item CUSOLVE: 10s 
    \item Sparse Julia \lstinline{J\F} (probably UMFPACK): 3.49s 
    \item Block Jacobi + BiCGSTAB in ExaPF.jl
    \begin{itemize}
      \item Jacobian: 14.2 ms
      \item Preconditioner: 357 ms
      \item BiCGSTAB: 4.42s
      \item Total: 4.8s
    \end{itemize}
  \end{itemize}
\end{frame}

% \begin{frame}
%   \frametitle{Results}
%   \begin{itemize}
%     \item Code runs on local workstation {\it moonshot} and Summit
%     \item NVIDIA Quadro GV100 based on Volta 
%   \end{itemize}
% \end{frame}

% \begin{frame}[fragile]
%   \frametitle{ExaPF.jl}
%   {\bf 30,000 bus system}
%   \begin{center}
%    \includegraphics[width=0.8\textwidth]{figures/timings}
%   \end{center}
% \end{frame}

% \begin{frame}
%   \frametitle{Hessian and Jacobian Coloring}
%   \begin{center}
%     \begin{figure}
%       \includegraphics[width=0.45\textwidth]{figures/directionsgpu}\footnote{\cite{simdicpp}}
%     \end{figure}
%   \end{center}
%   \begin{itemize}
%     \item Number of colors for 9,241 bus system: 76, Hessian size: 580,587$\times$580,587, compressed: 580,587$\times$76
%     \item GPU strategy: Effectively SIMD parallelize over colors/directions
%     \item Sweet spot on NVIDIA GV100: Chunks of 32 directions (see Figure)
%     \item 0.3ms per color
%     \item We don’t expect number colors to exceed 500 for final case
%   \end{itemize}
% \end{frame}

\begin{frame}
  \frametitle{Conclusions}
  {\bf Optimization on GPUs for NLPs}
  \begin{itemize}
    \item Reduced gradient method for OPF stalls
    \item Handle inequality constraints
    \item Leverage machinery for second-order reduced method
    \item Reduced method optimization solver and integration with JuMP (MOI) in Julia
    \item SIMD modeling framework
    \item Support for GPU Arrays on Intel and AMD GPUs
  \end{itemize}
\end{frame}

\begin{frame}[noframenumbering,plain,allowframebreaks]{References}
    \printbibliography[heading=none]
\end{frame}




%\bibliographystyle{unsrtnat}
\end{document}

