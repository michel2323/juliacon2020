% DO NOT COMPILE THIS FILE DIRECTLY!
% This is included by the other .tex files.
\begin{frame}
  \frametitle{Overview}
  \begin{itemize}
    \item Motivation
    \item Automatic Differentiation
    \item Preconditioner
    \item Linear Solver
    \item Results
  \end{itemize}
\end{frame}
\begin{frame}
  \frametitle{Upcoming Supercomputers}
    \begin{center}
      \includegraphics[width=.25\textwidth]{./figures/ecp} \\
      % \includegraphics[width=.75\textwidth]{./figures/go} 
    \end{center}
  \begin{columns}[T]
    \begin{column}{0.49\textwidth}
      \begin{center}
        {\bf Aurora}\\
        \includegraphics[width=0.75\textwidth]{./figures/aurora}
      \end{center}
    \end{column}
    \begin{column}{0.49\textwidth}
      \begin{center}
        {\bf Frontier}\\
        \includegraphics[width=0.75\textwidth]{./figures/frontier}
      \end{center}
    \end{column}
  \end{columns}
  \begin{columns}[T]
    \begin{column}{0.49\textwidth}
    \begin{itemize}
      \item Intel’s Xe compute architecture.
      \item $>$ 1 exaflops
    \end{itemize}
    \end{column}
    \begin{column}{0.49\textwidth}
      \begin{itemize}
        \item AMD EPYC processors and Radeon Instinct GPU
        \item 1.5 exaflops
      \end{itemize}
    \end{column}
  \end{columns}
\end{frame}

\begin{frame}
  \frametitle{Power System}
  \begin{columns}
    \begin{column}{0.45\textwidth}
      \includegraphics[width=\textwidth]{figures/slides.png}
    \end{column}
    \begin{column}{0.45\textwidth}
      \begin{center}
        % \includegraphics[width=0.8\textwidth]{figures/DampedSine.png}
      \end{center}
      \begin{itemize}
        \item Protect against contingency scenarios
        \item Demand is uncertain
        \item Generation is uncertain (solar, wind, water)
        \item Recent developments in renewable energies lead to less mass in generators decreasing inertia
        \item Less inertia worsens effects of uncertainty
      \end{itemize}
    \end{column}
  \end{columns}
\end{frame}

\begin{frame}
  \frametitle{ExaSGD: Optimizing Stochastic Grid Dynamics at Exascale
}
\begin{columns}
  \begin{column}{0.45\textwidth}
    \includegraphics[width=\textwidth]{./figures/generation}
  \end{column}
  \begin{column}{0.45\textwidth}
    \includegraphics[width=\textwidth]{./figures/ramping}
  \end{column}
\end{columns}
  \begin{center}
      \end{center}
  {\bf Motivation:}
  \begin{itemize}
    \item More renewable energy, more uncertainty
    \item Increase in renewable energy generation ramping
  \end{itemize}
  {\bf Goals:}
  \begin{itemize}
    \item Useful long-term planning model under higher uncertainty
    \item Use AC power flow (nonlinear functions) 
    \item Efficiently leverage exascale hardware in 2021
  \end{itemize}
\end{frame}

\begin{frame}
  \frametitle{Complex Networks}
  \includegraphics[width=\textwidth]{figures/complexn}
  \begin{itemize}
    \item Examples: traffic, Internet, pipelines, power grid
    \item Connection pattern is irregular but not random
    \item Different from most PDE problems (e.g. CFD)
    \item Larger cases tend to lead to ill-conditioning
  \end{itemize}
\end{frame}

\begin{frame}[fragile]
  \frametitle{Optimal Power Flow}
  \begin{itemize}
    \item {\bf Objective}
    \begin{itemize}
      \item Generation cost at generators:
      $ \minimize \sum^G_{i=1} c_i(P_{g_i})$, where $c_i$ is the cost of generator $g_i$
    \end{itemize}
    \item {\bf Constraints}
    \begin{itemize}
      \item Kirchhoff's law: What flows in must flow out (nonlinear, non-convex in ACOPF)
      \begin{align*}
        V_k e^{-j\theta_k} & \sum^{N}_{m=0} (G_{km} + jB_{km})V_m e^{j\theta_m} = P_k - j Q_k,\ k = 1, \dots, N \\
        \text{where}\\
        V_k &\text{ voltage magnitude at node } k\\
        \theta &\text{ voltage angle at node } k\\
        G_{km} + jB_{km}& \text{ element of nodal admittance matrix}\\
        P_k , Q_k &\text{ net real and reactive power entering and leaving node } k
      \end{align*}
      \item Line limits:
      $$ \theta^{min}_{nm} \leq \theta_n - \theta_m \leq \theta^{max}_{nm}$$
    \end{itemize}
  \end{itemize}
\end{frame}

\begin{frame}[fragile]
  \frametitle{Use Interior-Point for NLP}
  {\bf Solve}
  \begin{align*}
  &\minimize f(x)\\ 
  \text{with}&\\
  &g(x) \geq 0, \ i=1,\dots, m \\
  &c(x) = 0, \ j=1,\dots, l \\
  \end{align*}
  {\bf using Newton and barrier functions}
  \begin{align*}
  &\minimize \phi_\mu := f(x) - \mu \sum^m_{i=1} \ln g(x)\\ 
  \text{with}&\\
  &c(x) = 0, \ j=1,\dots, l 
  \end{align*}
  \begin{itemize}
    \item Barrier functions exacerbate ill-conditioning
  \end{itemize}
\end{frame}

\begin{frame}[fragile]
  \frametitle{Interior-Point and GPUs}
  {\bf Current State}
  \begin{itemize}
    \item Very ill-conditioned system (up $1e^{16}$)
    \item Requires indefinite sparse direct inertia revealing solver
    \item Ipopt only supports direct solver (exception: Pardiso)
    \item default MUMPS, preferred are HSL libary MA27, MA57 
    \item PIPS-NLP implements Schur complement for two-stage optimization
    \item HiOp mixed dense-sparse solver 
  \end{itemize}
\end{frame}


\begin{frame}[fragile]
  \frametitle{\sout{Optimal} Power Flow}
  \begin{itemize}
    \item {\bf \sout{Objective}}
    \begin{itemize}
      \item \sout{Generation cost at generators:
      $ \minimize \sum^G_{i=1} c_i(P_{g_i})$, where $c_i$ is the cost of generator $g_i$}
    \end{itemize}
    \item {\bf Constraints}
    \begin{itemize}
      \item Kirchhoff's law: What flows in must flow out (nonlinear, non-convex in ACOPF)
      \begin{align*}
        V_k e^{-j\theta_k} & \sum^{N}_{m=0} (G_{km} + jB_{km})V_m e^{j\theta_m} = P_k - j Q_k,\ k = 1, \dots, N \\
        \text{where}\\
        V_k &\text{ voltage magnitude at node } k\\
        \theta &\text{ voltage angle at node } k\\
        G_{km} + jB_{km}& \text{ element of nodal admittance matrix}\\
        P_k , Q_k &\text{ net real and reactive power entering and leaving node } k
      \end{align*}
      \item \sout{Line limits: $ \theta^{min}_{nm} \leq \theta_n - \theta_m \leq \theta^{max}_{nm}$}
    \end{itemize}
  \end{itemize}
\end{frame}

\begin{frame}[fragile]
  \frametitle{Power Flow}
  {\bf Nonlinear equations}
  \begin{itemize}
      \item Kirchhoff's law: What flows in must flow out (nonlinear, non-convex in ACOPF)
      \begin{align*}
        V_k e^{-j\theta_k} & \sum^{N}_{m=0} (G_{km} + jB_{km})V_m e^{j\theta_m} = P - jQ,\ k = 1, \dots, N \\
        \text{where}\\
        V_k &\text{ voltage magnitude at node } k\\
        \theta &\text{ voltage angle at node } k\\
        G_{km} + jB_{km}& \text{ element of nodal admittance matrix}\\
        P, Q &\text{ net real and reactive power entering and leaving node } k
      \end{align*}
      \item Use Newton-Raphson to solve nonlinear equations
  \end{itemize}
\end{frame}

\begin{frame}
  \frametitle{Implementation}
  {\bf Goals}
  \begin{itemize}
    \item Compute Jacobian using automatic differentiation
    \item Implement a preconditioner
    \item Implement a Krylov method
    \item No computation on the host in main loop, no data transfer
    \item All in Julia, no external calls if possible
  \end{itemize}
\end{frame}

\begin{frame}
  \frametitle{}
  \centering
  {\Huge Automatic Differentiation}
\end{frame}

\begin{frame}[fragile]
  \frametitle{Derivatives}
  {\bf Newton-Raphson}
  \begin{minipage}{0.8\textwidth}
    \begin{lstlisting}
    go = true
    while(go)
      dx .= jacobian(x)\f(x)
      x  .= x .- dx
      go = norm(f(x)) < tol ? true : false
    end
    \end{lstlisting}
   \end{minipage}
  {\bf Automatic Differentiation}
  \begin{itemize}
    \item \lstinline{F = f(x)} $\rightarrow$ \alert{\lstinline{J = jacobian(x)}}\footnote{\cite{RevelsLubinPapamarkou2016}}
    \item Two modes: \Colorbox{green}{Adjoint} or \Colorbox{red}{tangent}
    \item \Colorbox{green}{\lstinline{adj(x, y) = (J(x))'*y)}}, \Colorbox{red}{\lstinline{tgt(x,d) = J(x)*d}} 
    \item \Colorbox{green}{\lstinline{size(x) >> size(F)}} or \Colorbox{red}{\lstinline{size(x) <= size(F)}}
    \item number of buses $\propto$ \lstinline{size(x) = size(F)}
  \end{itemize}
\end{frame}

\begin{frame}
  \frametitle{Jacobian Coloring}
  \begin{center}
    \begin{figure}
      \includegraphics[width=0.65\textwidth]{figures/compression}\footnote{Thanks to Paul Hovland}
      \caption{Jacobian compression}
    \end{figure}
  \end{center}
  \begin{itemize}
    \item Using greedy algorithm in \lstinline{SparseDiffTools.jl}
    \item To get full Jacobian, one has to do $c$ times \lstinline{tgt(x,d) = J(x)*d}
    \item Complexity for Jacobian goes from $\bigo{n} \times cost(f)$ to $\bigo{c} \times cost(f)$
  \end{itemize}
\end{frame}

\begin{frame}
  \frametitle{AD on GPUs in Julia}
  \begin{center}
      \includegraphics[width=0.75\textwidth]{figures/F}
  \end{center}
\end{frame}
\begin{frame}
  \frametitle{AD on GPUs in Julia}
  \begin{center}
    \lstinline{F = T(undef, n)}
  \end{center}
  \begin{itemize}
    \item Float vector: \lstinline{T = Vector\{Float64\}}
    \item Arbitrary precision vector: \lstinline{T = Vector\{BigFloat\}}
    \item First-order tangent: \lstinline{t1s\{N\} =  ForwardDiff.Dual\{Nothing,Float64, N\} where N}
    \item Second-order tangent: \lstinline{t2s\{M,N\} =  ForwardDiff.Dual\{Nothing,t1s\{N\}, M\} where M, N}
    \item First-order tangent vector: \lstinline{T = Vector\{t1s\{N\}\}}
    \item First-order tangent GPU vector: \lstinline{T = CuVector\{t1s\{N\}\}}
    \item Relying on \lstinline{CUDA.jl} \footnote{\cite{besard2018juliagpu}} \footnote{\cite{besard2019prototyping}}
  \end{itemize}
\end{frame}

\begin{frame}
  \frametitle{AD on GPUs in Julia}
  \begin{columns}[T]
    \begin{column}{0.35\textwidth}
      \begin{center}
        \vspace{0.2cm}
        \includegraphics[width=\linewidth]{figures/compression}
      \end{center}
    \end{column}
    \begin{column}{0.6\textwidth}
      \begin{center}
        \includegraphics[width=\linewidth]{figures/simd}
      \end{center}
    \end{column}
  \end{columns}
  \begin{center}
  \end{center}
  {\bf SIMD over sparsity colors}
  \begin{itemize}
    \item Change from \lstinline{F = f(x)} to \lstinline{J = f(x)} on GPUs is a type change from \lstinline{Vector\{Float64\}} to \lstinline{CuVector\{t1s\{c\}\}}
  \end{itemize}
\end{frame}

\begin{frame}
  \frametitle{}
  \centering
  {\Huge Preconditioner}
\end{frame}

\begin{frame}
  \frametitle{Linear solver}
  {\bf Tour de Solvers, solver we have experience with}
  \begin{table}
  \begin{tabular}{p{5cm}|lll}
    MA27, MA57, MUMPS, SPQR, CUSOLVER, SPRAL SSIDS & indefinite & sparse & direct \\
    \hline
    BLAS, CUBLAS & indefinite & dense  & direct \\
    \hline
    CHOLMOD, SPRAL SSIDS& positive definite & sparse & direct \\
    \hline
    \lstinline{IterativeSolvers.jl}, \lstinline{Krylov.jl}, PETSc & indefinite & sparse & iterative \\ 
  \end{tabular}
\end{table}
  \begin{itemize}
    \item Best for GPU: Iterative
  \end{itemize}
\end{frame}

\begin{frame}
  \frametitle{Preconditioner}
  \begin{center}
    \includegraphics[width=0.25\textwidth]{figures/mpiblocks}
    \includegraphics[width=0.25\textwidth]{figures/gpublocks}
  \end{center}
  {\bf Block-Jacobi}
  \begin{itemize}
    \item GPU implementation requires a large number of blocks
    \item PETSc per MPI process: ILU + Krylov \footnote{\cite{schwarz}} 
    \item GPU: Batch CUBLAS LU inversion
    \item Expectation: Increase in number of blocks leads to worse preconditioner
  \end{itemize}
\end{frame}

\begin{frame}
  \frametitle{Preconditioner}
  {\bf Setup}
  \begin{itemize}
    \item Create a partitioning (METIS)
  \end{itemize}
  {\bf Update P}
  \begin{tabular}{m{0.15\textwidth}m{0.05\textwidth}m{0.3\textwidth}m{0.05\textwidth}m{0.3\textwidth}}
        \includegraphics[width=0.15\textwidth]{./figures/compressed}
    &
    {\Huge $\rightarrow$}
    &
        \includegraphics[width=0.3\textwidth]{figures/gpublocks}
    &
    {\Huge $\rightarrow$}
    &
        \includegraphics[width=0.3\textwidth]{./figures/csr}
    \\
  \end{tabular}
    \begin{enumerate}
      \item Read dense compressed Jacobian into dense Jacobi blocks
      \item Batch inversion
      \item Update sparse matrix P from dense Jacobi blocks
    \end{enumerate}
  {\bf Code size}
  \begin{itemize}
    \item 200 lines of code for BOTH CPU and GPU implementation
  \end{itemize}
\end{frame}

\begin{frame}
  \frametitle{Preconditioner}
  \includegraphics[width=0.95\textwidth]{figures/preconditioner}
\end{frame}

\begin{frame}
  \frametitle{}
  \centering
  {\Huge Linear Solver}
\end{frame}

\begin{frame}
  \frametitle{Linear solver}
  {\bf GMRES vs. BiCGSTAB}
  \begin{center}
   \includegraphics[width=0.7\textwidth]{figures/gmresbicgstab.png}
  \end{center}
  \begin{itemize}
    \item GMRES and BiCGSTAB \footnote{\cite{bicgstabVorst}} \footnote{\cite{sleijpen1993bicgstab}} in \lstinline{IterativeSolvers.jl}, no GPU support 
    \item GMRES in \lstinline{Krylov.jl}, no BiCGSTAB 
  \end{itemize}
\end{frame}


\begin{frame}[fragile]
  \frametitle{Linear Solver}
  \begin{center}
   \includegraphics[width=0.7\textwidth]{figures/bicgstab.png}
  \end{center}
\end{frame}

\begin{frame}
  \frametitle{SIMD on GPUs in Julia}
  \begin{center}
      \includegraphics[width=0.75\textwidth]{figures/F}
  \end{center}
\end{frame}
\begin{frame}
  \frametitle{AD on GPUs in Julia}
  \begin{center}
    \lstinline{F = T(undef, n)}
  \end{center}
  \begin{itemize}
    \item Float vector: \lstinline{T = Vector\{Float64\}}
    \item Arbitrary precision vector: \lstinline{T = Vector\{BigFloat\}}
    \item First-order tangent: \lstinline{t1s\{N\} =  ForwardDiff.Dual\{Nothing,Float64, N\} where N}
    \item Second-order tangent: \lstinline{t2s\{M,N\} =  ForwardDiff.Dual\{Nothing,t1s\{N\}, M\} where M, N}
    \item First-order tangent vector: \lstinline{T = Vector\{t1s\{N\}\}}
    \item First-order tangent GPU vector: \lstinline{T = CuVector\{t1s\{N\}\}}
    \item Relying on \lstinline{CUDA.jl} \footnote{\cite{besard2018juliagpu}} \footnote{\cite{besard2019prototyping}}
    \item \alert{broadcast operator \lstinline{x .= a .* b}}
  \end{itemize}
\end{frame}

\begin{frame}
  \frametitle{}
  \centering
  {\Huge Results}
\end{frame}

\begin{frame}[fragile]
  \frametitle{ExaPF.jl}
  \begin{lstlisting}
    pkg> instantiate
    julia> target = "cuda" 
    julia> include("examples/pf.jl") 
    julia> datafile = "test/case14.raw"
    julia> sol, conv, res = pf(datafile, 8, "bicgstab") # slow
    julia> sol, conv, res = pf(datafile, 8, "bicgstab")
  \end{lstlisting}
  \begin{itemize}
    \item \url{https://exanauts.github.io/}
    \item Julia for Summit builds
    \item Other research software
    \item \url{https://github.com/exanauts/ExaPF.jl}
  \end{itemize}
\end{frame}

\begin{frame}[fragile]
  \frametitle{ExaPF.jl}
  {\bf 30,000 bus system}
  \begin{lstlisting}
    julia> datafile = "GO-Data/.../Network_30R-025/.../case.raw"
    julia> sol, conv, res = pf(datafile, 1000, "bicgstab")
  \end{lstlisting}
  \begin{itemize}
    \item \lstinline{size(J)} = 57,721 $\times$ 57,721 
    \item Block Jacobi size: 59 $\times$ 59 $\approx$ 25 MB
    \item Total GPU memory usage: 5 GB
    \item Number of Jacobian colors: 29
    \item 4 Newton iterations with $tol = 1e^{-6}$, total BiCGSTAB iterations: 3,752 
  \end{itemize}
\end{frame}

\begin{frame}
  \frametitle{Results}
  \begin{itemize}
    \item Code runs on local workstation {\it moonshot} and Summit
    \item NVIDIA Quadro GV100 based on Volta 
  \end{itemize}
\end{frame}

\begin{frame}[fragile]
  \frametitle{ExaPF.jl}
  {\bf 30,000 bus system}
  \begin{center}
   \includegraphics[width=0.8\textwidth]{figures/timings}
  \end{center}
\end{frame}

\begin{frame}
  \frametitle{ExaPF.jl}
  {\bf Context}
  \begin{itemize}
    \item CUSOLVE: 10s 
    \item Sparse Julia \lstinline{J\\F} (probably UMFPACK): 3.49s 
    \item Block Jacobi + BiCGSTAB in ExaPF.jl
    \begin{itemize}
      \item Jacobian: 14.2 ms
      \item Preconditioner: 357 ms
      \item BiCGSTAB: 4.42s
      \item Total: 4.8s
    \end{itemize}
  \end{itemize}
\end{frame}

\begin{frame}
  \frametitle{Hessian and Jacobian Coloring}
  \begin{center}
    \begin{figure}
      \includegraphics[width=0.45\textwidth]{figures/directionsgpu}\footnote{\cite{simdicpp}}
    \end{figure}
  \end{center}
  \begin{itemize}
    \item Number of colors for 9,241 bus system: 76, Hessian size: 580,587$\times$580,587, compressed: 580,587$\times$76
    \item GPU strategy: Effectively SIMD parallelize over colors/directions
    \item Sweet spot on NVIDIA GV100: Chunks of 32 directions (see Figure)
    \item 0.3ms per color
    \item We don’t expect number colors to exceed 500 for final case
  \end{itemize}
\end{frame}

\begin{frame}[fragile]
  \frametitle{ExaPF.jl}
  {\bf BiCGSTAB iterations with decreasing block size}
  \begin{center}
   \includegraphics[width=0.45\textwidth]{figures/blocks}
   \includegraphics[width=0.45\textwidth]{figures/bicgstabiter}
  \end{center}
  \begin{itemize}
    \item Smaller blocks decrease memory footprint by $\frac{1}{n}$ 
    \item Smaller blocks speed up application of preconditioner in BiCGSTAB $P * x$
    \item BiCGSTAB unimpressed by smaller block sizes, however FAILS to converge without preconditioner
  \end{itemize}
\end{frame}

\begin{frame}
  \frametitle{Conclusions}
  \begin{itemize}
    \item Much harder systems for OPF with inequality constraints
    \item Other GPU first preconditioners
    \item Rapid prototyping in Julia is fast
    \item SIMD modeling framework
    \item Reduced gradient method for OPF
    \item Support for GPU Arrays on Intel and AMD GPUs
  \end{itemize}
\end{frame}

\begin{frame}[noframenumbering,plain,allowframebreaks]{References}
    \printbibliography[heading=none]
\end{frame}



